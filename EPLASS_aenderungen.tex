\documentclass{article}
\usepackage[ngerman]{babel}
\usepackage{lmodern}


\section{Organisation}
EPLASS hat den Wunsch geäußert, dass innerhalb der ABDSB eine zentraler Ansprechpartner für EPLASS installiert wird. Mit dem Wechsel von Frau Achhammer zum StBAR wird diese Funktion vom SG52 in Person von Herr Schmidl übernommen.

Darüberhinaus fugiert in jedem Sachgebiet noch ein Ansprechpartner, der die spezifischen Anforderungen innerhalb des Sachgebiets sammelt und weitergibt. Hierfür wurde folgender Personenkreis festgelegt:

\begin{itemize}
  \item SG51: Herr Hillebrand
  \item SG53: Herr Hänsel (zugleich Vertreter von Herrn Schmidl)
  \item SG54: Frau Stampehl
\end{itemize}

\section{Plannummernkonvention}
\begin{itemize}
  \item Implementierung Teilbauwerke\\
    Insbesondere bei Maßnahmen aus dem Bereich des Tunnelbaus besteht der Bedarf Teilbauwerke zu implementieren. Wobei es sich hier nicht um Teilbauwerke im Sinne der ASB handelt, sondern vielmehr einzelne Bauteile/Abschnitte des Gesamtbauwerks. Dies soll über eine größere Zifferntiefe im Rahmen der Plannummernkonvention erreicht werden. Hierbei werden die Festlegungen der Maßnahme Tunnel »Tutting« übernommen:
    \leftskip{Die Plannummer wird nach der ASB-Nr. um zwei Stellen für das Teilbauwerk/den Abschnitt ergänzt (z. B. T1)}
  \item Firmenpläne
    Die sog. »Firmenpläne«, also Planunterlagen, die idR. Baubehelfe beinhalten, also Leistungen für Bauzwecke, die später nicht am Bauwerk verbleiben,


\end{itemize}

\section{Projektdokumentation}
Sämtliche Standardprojekte aus dem Bereich des Ingenieurbaus sollen über eine gemeinsames Standarddokument abgebildet werden können. Für das Projekt nicht zutreffende Elemente werden dabei durch den Projektverantwortlichen als solche im Dokument kenntlich gemacht (z. B. durch Durchstreichen).





\section{Anpassung Workflow}
Die aktuell eingesetzen (Standard-) Workflows sollen um folgende Punkte ergänzt bzw. modifiziert werden:
\begin{itemize}
  \item BÜ-Leserechte schon bereits bei Workflow 03b (\emph{Aufgabe: vertragliche Prüfung}) implementieren
  \item Einbinden Repro-Service in den Workflow, erwirkt durch den Projektverantwortlichen per Aktivierung
  \item Implementierung Pläne Parken zu können
  \item Ermächtigung Poweruser die Prüffristen für ausgewählte Pläne/Planpakete bzw. auch global aussetzen zu können (\emph{Stichwort: Weihnachtspause})
  \item
\end{itemize}

\section{Planköpfe}

\section{Userberechtigungen}
