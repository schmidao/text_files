%        File: sem_projektmanagement.tex
%     Created: Mo Feb 12 01:00  2018 Mitteleuropäische Z
% Last Change: Mo Feb 12 01:00  2018 Mitteleuropäische Z
%
\documentclass[a4paper]{article}
\usepackage[german]{babel}
\usepackage[utf8]{inputenc}
\usepackage[T1]{fontenc}
\usepackage{lmodern}

\begin{document}

\usepackage{scrartcl}

\section{Allgemenin}
\begin{itemize}
  \item ``Wer zahlt schaft an.'' Ja, aber es geht  darum, die Konsequenzen des Anschaffens deutlich zu machen:
    \begin{itemize}
      \item eine gute, machbare Lösung wäre \ldots
      \item Ja, das geht, wenn \ldots
      \item folgendes ist unverzichtbar \ldots
      \item was wir inhaltlich / zeitlich strecken können \ldots
      \item nur Argumente bringen, die auch für die Gegenseite relevant sind:\\
	Termin, Qualität, Kosten, Risiko, Beistellleistung.
    \end{itemize}
  \item Verteidigungsstrategie:\\
    Ich habe rechtzeitig alles getan, was ich innerhalb meiner Befugniss tun konnte um mein Projekt erfolgreich zu machen.\\
    Ich habe rechtzeitig einen Vorschlag gemacht, was darüber hinaus geschehen müsste.\\
    Alle dafür wichtigen Personen kennen diesen Vorschlag.\\
    Alle wichtigen Personen sind informiert über die Dringlichkeit und die Folgen.
  \item Nutzen für Auftraggeber und Auftragnehmer: worin besteht der größte Nutzen für beide Seiten?
    \begin{itemize}
      \item worauf sollte man sich dementsprechend konzentrieren?
    \end{itemize}
  \item Es gibt vier Stellschrauben, um Ziele zu erreichen: Termin, Leistungszugang, Ressourcen und Qualität. Darüberhinaus kann ``anders'' gearbeitet werden (siehe Projektdurchführung und Projektsteuerung).
  \item Führen durch Fragen
  \item Oft kann im vornherein nicht gesagt werden, was ``richtig'' oder ``gut'' ist. Das wird sich erst im Laufe des Projektes herausstellen. Es geht vielmehr darum, was``nützlich oder nicht nützlich'' zur Erreichung der Projektziele ist.
  \item ``Gib mir den Mut, Dinge zu ändern, die ich ändern kann,\\
    die Gelassenheit, Dinge hinzunehmen, die ich nicht ändern kann\\
    und die Weisheit, das eine vom anderen zu unterscheiden!''
\end{itemize}

\section{Auftragsklärung / Projektdefinition}
\begin{itemize}
  \item Übergeordneten Sinn verstehen, Auslöser und Schwerpunkte klären.
  \item Erwartungen aufnehmen.
  \item Ziele priorisieren und zur aktiven Steuerung verwenden:
    \begin{itemize}
      \item welches Minimum an Leisttungsumfang ist erforderlich, die Kernziele zu erreichen?
      \item im Leistungsumfang (in scope) / nicht im Leistungsumfang (out of scope)
      \item Leistungsstufen in Betracht ziehen.
    \end{itemize}
  \item Stakeholder identifizieren und Strategie entwickeln.
  \item Erfolgsfaktoren / Risiken identifizieren und Projektstrategie darauf abstimmen.
  \item K U S -- Struktur (Klar / Unklar / Strittig) als Gesprächsleitfaden.
  \item Ergebnisse des Auftragsgesprächs unverzüglich dokumentieren und an den Auftraggeber wiederspiegeln, schriftliches Feedback einholen, zeitnahen Folgetermin vereinbaren, um weitere Fragen zu klären.
\end{itemize}

\section{Projektaufbauorganisation}
\begin{itemize}
  \item Rollen und Verantwortungen klar regeln, Erwartungen klären.
  \item Einbeziehung der Beteiligten ist einer der wichtigsten Erfolgsfaktoren in Projekten.
  \item Auch das ``Management'' ist ``Ressource'' für die Projektleitung
  \begin{itemize}
    \item für einen funktionierenden Lenkungsausschuss sorgen.
  \end{itemize}
  \item Eskalation ist ein "normales" Werkzeug des Projektmanagements.
  \item Spielregeln festlegen.
\end{itemize}

\section{Projekt(management)planung}
\begin{itemize}
  \item Projektplanung ist in erster Linie eine Kommunikationsbasis. Zahlen -- Daten -- Erfolgsfaktoren
  \item "wenn wir in diesen Rahmenbedingungen arbeiten, ist auf Basis unserer Erwartungswerte folgende Entwicklung am wahrscheinlichsten ..."
  \item Vom Ergenis her planen: was kann ich tun, um das Gewünschte zu erreichen?
  \item Vom Termin her in die Phasen planen, Arbeitspakete identifizieren mit Organisationseinheit und Aufwand, Arbeitspakete "einpassen" in Phasen, Ressourcen verhandeln (wenn -- dann).
  \item Aufwandsschätungen: Annahmen und Randbedingungen hinterfragen, Zwei-Punkt-/ Drei-Punkt-Schätzungen, Planning Poker.
  \item Stakeholdermanagement-Strategien und Risiko-Management-Maßnahmen berücksichtigen
\end{itemize}

\section{Projektstart}
\begin{itemize}
  \item ``Sage mir, wie ein Projekt startet, und ich sage Dir, wie es endet.''
    \begin{itemize}
      \item Drei Veranstaltungen: allgemeine Information, erster Lenkungsausschuss, Team-Kick-Off.
    \end{itemize}
  \item Team-Kick-Off: gemeinsamer Start, an dem alle zusammenkommen, anschließend informeller Teil, in dem sich alle persönlich kennen lernen. Vorgehen, Organisation, Arbeitsweise klären.
  \item End-to-end Verantwortung implementieren.
  \item Mit dem ersten Eindruck / den ersten Arbeiten werden de facto die ``Spielregeln'' geschaffen.
\end{itemize}

\section{Projektsteuerung}
\begin{itemize}
  \item Aktive Steuerung, keine bloße Verfolgung
  \begin{itemize}
    \item Trendanalyse, Vorschlag Steuerungsmaßnahmen
    \item "um den Erfolg des Vorhabens zu sichern, empfehle ich \ldots"
    \item "uns würde es helfen, wenn \ldots"
  \end{itemize}
  \item Prognose -> "wenn wir so weiter machen, ist auf Basis unserer Erfahrungen folgende Entwicklung am wahrscheinlichsten \ldots"
  \item Nachhaken -> "Unter welchen Voraussetzungen sehen Sie sich in der Lage, \ldots"
\end{itemize}

\end{document}
